\documentclass[a4paper,12pt,french,draft]{report}
\usepackage{babel} 
\usepackage[T1]{fontenc} 
\usepackage[utf8]{inputenc} 
\usepackage{lmodern}
\usepackage{amsthm}
\usepackage{stmaryrd}
\usepackage{amsmath}
\usepackage{amssymb}
\usepackage{mathrsfs}
\usepackage{geometry}
\usepackage{graphicx}
\usepackage{boiboites}
\geometry{hmargin=2.5cm,vmargin=2cm}
%\newtheorem{axiome}{Axiome}[section]
\newtheorem{lemma}{Lemme}[section]
%\newtheorem{proposition}{Proposition}[section]
%\newtheorem{theorem}{Théorème}[section]
\newtheorem{definition}{Définition}[section]
\newtheorem{exemple}{Éxemple}[section]
\newtheorem{corollaire}{Corollaire}[section]

% ------

\definecolor{violet}{rgb}{0.5,0,0.5}
\definecolor{orange}{rgb}{0.9,0.4,0}

\newboxedtheorem[boxcolor=black,titleboxcolor = black,titlebackground=blue!20,titlecolor = black]{proposition}{Proposition}{subsection}
\newboxedtheorem[boxcolor=red,titleboxcolor = red,titlebackground=red!20,titlecolor = black]{axiome}{Axiome}{}
\newboxedtheorem[boxcolor=blue,titleboxcolor = blue,titlebackground=blue!20,titlecolor = black]{theorem}{Théorème}{}

% ------


\title{TIPE : Origami et constructibilité}

\begin{document}
\newtheorem*{remarque}{Remarque}
\maketitle
\renewcommand{\contentsname}{Sommaire}
\tableofcontents{}

% --------------------- -------- --------------------- 
% --------------------- PARTIE I --------------------- 
% --------------------- -------- --------------------- 

\chapter{Introduction à la constructibilité : règle et compas}
		Il est nécessaire de commencer notre étude en posant quelques définitions et résultats élémentaires qui nous seront utiles par la suite.

% ----------------------------------------------------
		\section{Outils d'algèbre}
			
			\begin{definition}[Corps] 
				Un corps est une structure algébrique \( (\mathbb{K} , + , \times ) \) telle que :
				\begin{enumerate}
					\item Les lois \( + \) et \( \times \) sont des applications \( \mathbb{K}^2 \longrightarrow \mathbb{K} \)
					\item \((\mathbb{K} , +)\) soit un groupe commutatif, c'est-à-dire :
						\[ 
							\left\{ 
							\begin{array}{lll}
								\exists \, 0_{\mathbb{K}} \in \mathbb{K} , \forall x \in \mathbb{K}, x + 0_{\mathbb{K}} = 0_{\mathbb{K}} + x = x
								\\
								\forall x \in \mathbb{K} , \exists k \in \mathbb{K} , x + k = k + x = 0_{\mathbb{K}} \mbox{  et on notera } -x = k
								\\
								\forall (x,y) \in \mathbb{K}^{2},
											x + y = y + x
										
								\\
								\forall (x, y, z) \in \mathbb{K}^3, (x + y) + z = x + (y + z)
									 
							\end{array}
							\right.
						\]
					\item \((\mathbb{K}\setminus\{0\}, \times) \) est un groupe commutatif :
						\[ 
								\left\{ 
								\begin{array}{lll}
									\exists \, 1_{\mathbb{K}} \in \mathbb{K}\setminus\{0\} , \forall x \in \mathbb{K}\setminus\{0\}, x \times 1_{\mathbb{K}} = 1_{\mathbb{K}} \times x = x
									\\
									\forall x \in \mathbb{K}\setminus\{0\} , \exists k \in \mathbb{K}\setminus\{0\} , x \times k = k \times x = 1_{\mathbb{K}} \mbox{  et on notera } x^{-1} = k
									\\
									\forall (x,y) \in (\mathbb{K}\setminus\{0\})^{2},
												x \times y = y \times x
									\\
									\forall (x, y, z) \in (\mathbb{K}\setminus\{0\})^3, (x \times y) \times z = x \times (y + z)
										 
								\end{array}
								\right.
							\]
					\item La loi \( + \) est distributive sur la loi \( \times \) :
						\[
						\forall (x, y, z) \in \mathbb{K}^3, x \times (y + z) = (y + z) \times x = xy + xz
						\]
				\end{enumerate}
			
			\end{definition}
			
			\begin{definition}[Corps engendré]
				Soit \(\mathbb{E}\) et \(\mathbb{F}\) deux corps, tel que \(\mathbb{E} \subset \mathbb{F}\), et soit \(x \in \mathbb{F} \). Le corps engendré par \(x\), noté \(\mathbb{E}(x)\), est le plus petit corps (au sens de l'inclusion) qui contient x et \(\mathbb{E}\).
			\end{definition}
			
			\begin{definition}[Sous anneau engendré par x]
				Soit \(\mathbb{E}\) et \(\mathbb{F}\) deux corps, tels que \(\mathbb{E}\subset\mathbb{F}\), et \(x\in\mathbb{F}\). Alors on appelle \emph{Sous anneau de F engendré par x} l'ensemble :
				\[\left\{y\in F / \exists P \in E[X], y = P(x)\right\}\]
				On le note \(E[x]\).{}
				
				\begin{proof}
					\(E[x]\) est l'image de l'anneau \(E[X]\) par le morphisme d'évaluation \[\varphi_{x} : \; P \mapsto P(x) \] donc c'est un anneau. Par la suite on introduira \(\varphi_{x}\) systématiquement, sans le redéfinir.
				\end{proof}
				
			\end{definition}
			
			
			\begin{definition}[Espace Vectoriel]
				Un \(\mathbb{K}\)-espace vectoriel \(E\)  est une structure algébrique \( (E, + , \cdot)\) telle que
				\begin{enumerate}
					\item \( (E, +) \) est un groupe commutatif
					\item La loi externe \( \cdot \) est une application \( \mathbb{K} \times E \longrightarrow E \) qui vérifie les axiomes suivants :
						\begin{enumerate}
							\item Pseudo-associativité :
								\(
								\forall (\lambda, \mu, x) \in \mathbb{K}^2 \times E, \mu(\lambda x) = (\mu \lambda) x
								\)
							\item Pseudo-distributivité :
								\(
								\left\{
									\begin{array}{ll}
										\forall (\lambda, \mu, x) \in \mathbb{K}^2 \times E , (\lambda + \mu)x = \lambda x + \mu x \\
										\forall (\lambda, x, y) \in \mathbb{K} \times E^2 , \lambda(x + y) = \lambda x + \lambda y
									\end{array}
								\right.
								\)
							\item Opérateur neutre :
								\(
								\forall x \in E, 1_\mathbb{K} \cdot x = x
								\)
						\end{enumerate}
				\end{enumerate}
				
			\end{definition}
			
			
			\begin{definition}[Extension de corps]
			
				Si \( \mathbb{K} \) et \( \mathbb{L} \) sont deux corps tels que \(\mathbb{K} \subset  \mathbb{L} \), alors
				\(\mathbb{L}\) est un \(\mathbb{K}\)-espace vectoriel. Dès lors, \(\mathbb{L}\) est appelée une \emph{extension} de \(\mathbb{K}\).
			
				\begin{proof}
					La structure d'espace vectoriel de \(\mathbb{L}\) s'hérite de sa structure de corps.
					\begin{enumerate}
						\item \(\mathbb{L}\) est un corps donc, en particulier, \((\mathbb{L}, +)\) est un groupe commutatif.
						\item Tout élément de \(\mathbb{K}\) étant en particulier un élément de \(\mathbb{L}\), il suffit de définir la loi
						externe \(\cdot\) à l'aide de la loi \( \times \) de \(\mathbb{L}\) pour que tous les axiomes de la loi externe d'un espace vectoriel s'héritent directement des propriétés de la loi \( \times \) d'un corps:
							\[
							\forall (x, y) \in \mathbb{K} \times \mathbb{L}, x \cdot y = x \times y
							\]
					\end{enumerate}
				\end{proof}
			\end{definition}
			
			
			
			
			\begin{definition}[Degré d'une extension]
				Soit \(\mathbb{L}\) une extension d'un corps \(\mathbb{K}\). Si la dimension de \(\mathbb{L}\) en tant que \(\mathbb{K}\)-espace vectoriel est finie, on l'appelle le \emph{degré} de l'extension, notée :
				\[
				[\mathbb{L}:\mathbb{K}] = \dim_\mathbb{K}\mathbb{L}
				\]
				S'il est égal à 2, on parlera d'\emph{extension quadratique}.
			\end{definition}
			
			\begin{proposition}[Relation de Chasles sur le degré]
				Soit \(K\), \(L\), \(M\) trois corps tels que \( K \subset L \subset M\), et les extensions sont de degré fini. On a alors :
				\[
				[M:K] = [M:L]{\times} [L:K]
				\]
			\end{proposition}
				\begin{proof}
					On pose tout d'abord :
					\[
					\left\{
						\begin{array}{ll}
							p = [M:L] \\
							n = [L:K]
						\end{array}
					\right.
					\]
					
					
					Ensuite, exhibons une base de \(M\) en tant que \(K\)-espace vectoriel afin d'obtenir sa dimension.
					Soit \((l_i)_{1 \leq i \leq n} \in L^n\) une base de \(L\) en tant que \(K\)-ev.
					Soit \((m_j)_{1 \leq j \leq p} \in L^n\) une base de \(M\) en tant que \(L\)-ev.
					
					Montrons que \((l_i m_j)_{(i, j) \in \llbracket 1, n \rrbracket \times \llbracket 1, p \rrbracket}\) est une base de M en tant que \(K\)-espace vectoriel.
					\begin{enumerate}
					\item \emph{Caractère générateur} : soit \(x \in M\), alors 
						\[
						\exists (\lambda_1, \dots, \lambda_p) \in L^p, \sum_{j  = 1}^{p} \lambda_j m_j = x
						\]
						Or \(\forall i \in \llbracket 1, p \rrbracket, \lambda_i \in L\), donc :
						\[ 
						\exists (\mu_{i, 1}, \dots, \mu_{i, n}) \in K^n, \sum_{j  = 1}^{n} \mu_{i, j} l_j = \lambda_i 
						\]
						Puis :
						\[
						x 	= \sum_{i  = 1}^{p} \lambda_i m_i  
							= \sum_{i  = 1}^{p} \left(\sum_{j  = 1}^{n} \mu_{i, j} l_j\right) m_i 
							= \sum_{(i, j) \in \llbracket 1, p \rrbracket \times \llbracket 1, n \rrbracket}\mu_{i, j} (l_j m_i)
						\]
					\item \emph{Caractère libre} : soit \((\lambda_{i, j})_{ (i, j) \in \llbracket 1, n \rrbracket \times \llbracket 1, p \rrbracket} \in K^{np}\) telle que :
							\[
							\sum_{(i, j) \in \llbracket 1, n \rrbracket \times \llbracket 1, p \rrbracket} \lambda_{i, j} (l_j m_i) = O
							\]
						Soit encore :
							\[
							\sum_{i=1}^{p} \underbrace{\left( \sum_{j = 1}^{n} \lambda_{i, j} l_j \right)}_{\in L}m_i = 0
							\]
						Donc, par liberté de \((m_1, \dots, m_p)\), on a :
							\[
							\forall i \in \llbracket 1, p \rrbracket, \sum_{j = 1}^{n} \underbrace{\lambda_{i, j}}_{\in K} l_j = 0
							\]
						Donc, par liberté de \((l_1, \dots, l_n)\), on obtient:
							\[
							\forall i \in \llbracket 1, p \rrbracket, \forall j \in \llbracket 1, n \rrbracket, \lambda_{i, j} = 0
							\]
					\item On a ainsi montré que \((l_i m_j)_{(i, j) \in \llbracket 1, n \rrbracket \times \llbracket 1, p \rrbracket}\) est une base du \(K\)-espace vectoriel \(M\). Or, cette base contient \(n\times p\) vecteurs, donc l'extension M est bien de degré \(n\times p\) par rapport à \(K\), Ce qui achève la preuve. \( \)
					
					\end{enumerate}
				\end{proof}
    
    

% ----------------------------------------------------
		\section{Nombres algébriques}
			\subsection{Cas Général}
			\begin{definition}[Nombre algébrique]
				Soit \(E\), \(F\), deux corps,  et \(x \in F\). On dit que \(x\) est \emph{algébrique} sur E si : 
				\[
					\exists P \in E[X]\setminus\{0\} , P(x) = 0
				\]
				Sinon, on dit que x est \emph{transcendant}
			\end{definition}
			
			\begin{exemple}
			$ $
			    \begin{enumerate} 
				    \item Le nombre \(\sqrt{2}\) est algébrique sur $\mathbb{Q}$, car il est racine du polynôme \(X^{2} - 2\)
				    \item Le nombre $i$ est algébrique sur $\mathbb{Q}$, car il est racine du polynome \(X^{2} +1\)
			    \end{enumerate}
			\end{exemple}
			\begin{definition}[Degré algébrique et polynôme minimal]
				Soit \(x \in F \) algébrique sur \(E\). Alors il existe un unique polynôme unitaire de degré minimal, non nul, dont x soit racine. On l'appelle le \emph{polynôme minimal}, et son degré est le \emph{degré algébrique} de \(x\).{}
			\end{definition}
				\begin{proof}
					Soit \(x \in F\), algébrique sur \(E\). On remarque que l'ensemble des polynômes qui s'annulent en \(x\) est exactement \(Ker \varphi_{x}\) qui est un sous groupe additif. De plus, 
					\[
						\begin{aligned}
							\forall (P,Q) \in Ker\varphi_{x} \times E[X], \; \varphi_{x}(PQ) &= \varphi_{x}(P)\varphi_{x}(Q) \\
											&=P(x)Q(x) \\
											&= 0
						\end{aligned}
					\]
					Donc \(Ker \varphi_{x}\) est absorbant pour la deuxième loi, donc c'est un idéal. Or \(E[X]\) est principal, donc il existe un unique polynôme unitaire \(P \in E[X]\) tel que \(Ker \varphi_{x} = P \cdot E[X]\)\\
					Comme \(x\) est algébrique, il existe un polynôme non nul qui s'annule en \(x\) donc \(Ker \varphi_{x}\) n'est pas réduit à \(\{0\}\), donc \(P \neq 0_{E[X]}\) \\
					Enfin, soit \(Q \in E[X]\setminus\{0\}\) tel que \( Q(x) = 0 \), alors \(Q \in Ker \varphi_{x} = P \cdot E[X] \) donc \(P | Q\), ce qui assure que le degré de P est minimal.\\
					Si de plus \(Q\) est supposé de degré minimal et unitaire, on a \(\exists \lambda \in E\setminus\{0\} , Q = \lambda P \) (la non-nulité vient du fait que \(Q\) est lui même non nul). Donc \(P\) et \(Q\) sont associés, et par unicité du polynome unitaire dans une classe d'association, on a \[P = Q\] 
				\end{proof}
			
			\begin{proposition}
				Soit E un corps, F un sur-corps de E. Soit L une E-algèbre contenant E, inclut dans F, de dimension finie. Si \(x \in L\), alors x est algébrique sur E. En particulier, le résultat est vrai si L est une extension de corps
			\end{proposition}
				\begin{proof}
					Posons \(n = dim_{E}(L) \in \mathbb{N} \). Soit \(x \in L \). Considérons la famille
					\[{}
						\left( 1 , x , x^{2} , \dots , x^{n} \right)
						\]
					\begin{enumerate}
						\item Soit certains éléments apparaissent plusieurs fois, donc la famille est liée
						\item Soit tous les éléments sont différents, et comme famille a \(n+1\) éléments dans un espace de dimension \(n\), la famille est liée.
					\end{enumerate}
					Dans tous les cas, la famille est liée et on a : 
						\[{}
							\exists (\lambda_{0},\dots,\lambda_{n}) \in (E^{n})\backslash {(0,\dots,0)}, \sum_{i = 0}^{n} \lambda_{i}x^{i} = 0
						\]
						On pose alors : \(P = \sum_{i = 0}^{n} \lambda_{i}X^{i} \) qui n'est pas le polynome nul, car \( (\lambda_{0},\dots,\lambda_{n}) \) n'est pas la famille nulle. Et on a \(P(x) = 0 \), donc x est algébrique  \( \)
				\end{proof}
			
			\begin{proposition}
				Soit \(x \in F \), algébrique sur \(E\), alors \(E(x)\) est de dimension finie sur \(E\), et le degré de cette extension est égal au degré algébrique de \(x\){}
			\end{proposition}
				\begin{proof}
					Soit \(x \in F \), algébrique sur \(E\).{}
					
	\begin{enumerate}
		\item Montrons tout d'abord que $E[x]$ est une E-algèbre de dimension finie.\\ 
						Soit \(P_{0} \in E[X]\setminus\{0\} \) le polynôme minimal de $x$, notons \(n = deg(P_{0})\).
					Posons : 
					\[{}
						\begin{array}[t]{lccl}
							\varphi : 
							& E_{n-1}[X] & \longrightarrow & E[x] \\
							& P & \longmapsto & P(x)
						\end{array}
					\] et montrons que c'est un morphisme bijectif.
					
					\begin{enumerate}
						\item $\varphi$ est un morphisme comme restriction du morphisme d'évalutation $\varphi_{x}$
						\item \emph{Injectivité} - Soit \(P \in Ker(\varphi)\) et supposons que \(P \neq 0_{E_{{n_-1}}[X]} \). Quite à considérer le coefficient dominant de P, noté \(p\), et le polynôme \(\frac{1}{p}P\) unitaire, on peut supposer sans perte de généralité que P est unitaire. Comme \(P\) appartient à \(E_{n-1}[X]\), on a \(deg(P) < n\). Or \(P(x) = 0\), donc P contredit le choix de \(P_{0}\), ce qui est absurde. Donc P est le polynome nul, et \(\varphi\) est bien injectif
						
						\item \emph{Surjectivité}
						Soit $y \in E[x]$, on a : $\exists A \in E[X], A(x) = y$. Par division euclidienne : 
						\[{}
							\exists (Q,R) \in E[X] \times E_{n-1}[X] , A = PQ + R
						\]
						Donc, on a :
						\[{}
						\begin{aligned}
							y &= A(x) \\
							&= (PQ + R)(x)\\
							&=\underbrace{P(x)}_{=0}Q(x) + R(x)\\
							&=\varphi(R)
						\end{aligned}
						\]
						Donc $\varphi$ est bien surjective
					\end{enumerate}
				Finalement, par propriété des isomorphismes, on déduit que $E[x]$, comme espace isomorphe à un espace de dimension finie, est lui même de dimension finie, et :
				\[{}
				dim(E[x]) = dim(E_{n-1}[X]) = n = deg(P_{0})
				\]
		
		\item Montrons maintenant que $E[x]$ est un corps. Pour cela, il suffit de montrer que tout élément non nul de  $E[x]$ est invesible, les autres propriétés sont directement héritées de la structure d'algèbre.
			Soit $y \in E[x]\setminus\{0\} $. Comme élément d'une E-algèbre de dimension finie, par la proposition précédente, $y$ est algébrique sur $E$. Soit \(P_{1} \in E[X]\) son polynôme minmal, notons \(P = \sum_{i=0}^{m}a_{i}X^{i}\).{}
			\begin{enumerate}
				\item Supposons $a_{0} = 0$ , alors on aurait : $P_{1} = X\times\sum_{i=1}^{n}a_{i}X^{i-1}$ donc comme $y \neq O$, par intégrité,
				\[{}
					\sum_{i=1}^{n}a_{i}y^{i-1} = 0
				\]
				Donc $P_{1}$ ne serait pas de degré minimal, absurde. Donc : $a_{0} \neq 0 $
			
				\item On a en factorisant par y :
				\[{}
					\begin{aligned}
						P_{1}(y) = 0 &\Rightarrow \sum_{i=1}^{n}a_{i}y^{i} = -a_{0}\\
									&\Rightarrow \sum_{i=1}^{n}a_{i}y^{i-1} = \frac{-a_{0}}{y}\\
									&\Rightarrow \underbrace{\frac{-1}{a_{0}}\sum_{i=1}^{n}a_{i}y^{i-1}}_{\in E[x]} = \frac{1}{y}
					\end{aligned}
				\]
			\end{enumerate}
			Ainsi $y$ est inversible, et $E[x]$ est un corps.
			
		\item Montrons finalement que $E[x] = E(x)$. 
			\begin{enumerate}
				\item On a comme $E(x)$ est un corps, il est stable par produit, combinaisons linaires, donc \(\forall P \in E[X], P(x) \in E(x)\). Or \[ \forall y \in E[x], \exists P_{y} \in E[X], P(x) = y \] Donc,  \[E[x] \subset E(x)\]
				\item 
				Par construction de $E(x)$, tout corps contenant $E$ et $x$ contient \(E(x)\), donc, 
				\[ E[x] \supset E(x) \]
			\end{enumerate}
	\end{enumerate}
	Finalement \(E(x) = E[x]\), et c'est bien une extension de corps, de dimension finie, du degré algébrique de x.
				\end{proof}
				
		
		\begin{proposition}
			Soit $E,F$ deux corps, et $x,y \in F$, algébriques sur E. Alors, 
			$x+y$ et $xy$ sont aussi algébriques sur E. Enfin si $x\neq0$, $\frac{1}{x}$ est algébrique sur $E$. 
		\end{proposition}
		
		\begin{proof}
			Soit \(x,y \in F\), algébriques sur $E$. On note \(P_{x},P_{y} \in E[X]\) leur polynômes minimaux.
			\begin{enumerate}
				\item
				    Comme on a vu, $E(x)$ est un corps, donc il est légitime de considérer $(E(x))(y)$.
				    De plus $y$ est algébrique sur $(E(x))(y)$, car \(P_{y} \in E[X] \subset (E(x))[X]\), donc $(E(x))(y)$ est de dimension finie sur $E(x)$.
				\item 
				    Soit \[\mathcal{B} = (e_{1}, \cdots, e_{n}) \] une base de $E(x)$ (comme $E$-espace vectoriel), et \[\mathcal{C} = (\epsilon_{1},\cdots, \epsilon_{m})\] une base de $(E(x))(y)$ (comme $E(x)$-espace vectoriel).\\
				    Soit \(a \in (E(x))(y) \), on a : \[\exists (\lambda_{1},\cdots,\lambda_{m}) \in (E(x))^{m}, \; a = \sum_{i=1}^{m}\lambda_{i}\epsilon_{i}\]
				    Or comme $\mathcal{B}$ est en particulier génératrice dans $E(x)$, on a:
				    \[\forall i \in \llbracket 1 , m \rrbracket, \exists (\alpha_{i,1},\cdots,\alpha_{i,n}) \in E^{n}, \lambda_{i} = \sum_{j=1}^{n}\alpha_{i,j}e_{j} \]
				    Donc, finalement, 
				    \[a = \sum_{i=1}^{m} \left(  \sum_{j=1}^{n}\alpha_{i,j}e_{j} \right)\epsilon_{i} \]
				    Ainsi la famille \((e_{j}\epsilon_{i})_{(j,i) \in \llbracket 1 , n \rrbracket \times \llbracket 1 , m \rrbracket} \) est génératrice dans $(E(x))(y)$ vu comme {$E$-espace vectoriel}.
				\item{}
				    Or cette famille est finie, donc, $(E(x))(y)$ est de dimension finie sur $E$.  Or, \\\({x+y,xy \in (E(x))(y)}\), par la proposition précédemment démontrée, $x+y$ et $xy$ sont algébriques sur E. Enfin si $x \neq 0, \frac{1}{x} \in (E(x))(y)$ est aussi algébrique sur $E$.
			\end{enumerate}
		\end{proof}
		
		
		\begin{proposition}
			Les nombres algébriques sur E forment un corps.
		\end{proposition}
		\begin{proof}
		    C'est immédiat avec les résultats précédents.
		\end{proof}
	
		\subsection{Application à $\mathbb{Q}$}
	En ayant mis en place tous les outils nécessaires, on peut désormais aboutir à quelques résultats plus spécifiques à $\mathbb{Q}$, qui seront utiles pour établir avec précision la condition nécessaire et suffisante pour qu'un nombre soit constructible.
		
		\begin{proposition}
			Soit E une extension quadratique sur $\mathbb{Q}$. Alors
			\[{}
				\exists\, \delta \in \mathbb{Q}, E = \mathbb{Q}(\sqrt{\delta})
			\]
		\end{proposition}
		 Par la suite, on notera indifferemment $\mathbb{Q}(\sqrt{\delta})$ ou $\sqrt{\delta}\,\mathbb{Q}$
		\begin{proof}
			Soit E une extension quadratique sur $\mathbb{Q}$. Complétons la famille libre $(1_{\mathbb{Q}})$ en une base : $(1_{\mathbb{Q}},e)$. Comme élément d'une extension finie, $e$ est algébrique, on note $d$ son degré algébrique.
			\begin{enumerate}
				\item  Montrons que $d = 2$
				\begin{enumerate}
				    \item On a que $d\geq2$, car sinon $e$ serait racine d'un polynome de la forme $aX +b$, avec $(a,b) \in \mathbb{Q}^{2}$, donc serait rationnel. Or  $e \notin Vect(1_{\mathbb{Q}}) = \mathbb{Q}$ par liberté de la base. Donc \[d\geq 2\]
				    \item On a que $\mathbb{Q}(e) \subset E $ car une extension quadratique est en particulier un corps. En observant les dimensions, on a $d \leq dim_{\mathbb{Q}}(E) = 2$. Donc 
				    \[d \leq 2\] On a ainsi montré que $d=2$, et l'inclusion $\mathbb{Q}(e) \subset E $ avec l'égalité des dimensions donne : $E = \mathbb{Q}(e)$.
				\end{enumerate}
				
				\item  Soit $P = aX^{2} + bX +c \in \mathbb{Q}[X]$ le polynôme minimal de $e$. On a $a\neq 0$ sans quoi $e$ ne serait pas de degré algébrique 2. Donc \[e = \frac{-b \pm \sqrt{b^{2} - 4ac}}{2a}\]
				Posons $\delta = b^{2} - 4ac$ et montrons que $E = \mathbb{Q}(\sqrt{\delta})$.\\
				On a immédiatement que la famille \((1_{\mathbb{Q}},\sqrt{\delta})\) est libre, sans quoi $\sqrt{\delta}$ serait rationnel et $e$ aussi. Donc c'est une base, car E est de dimension 2.\\
				En réutilisant le résonnement du 1), avec $e' = \sqrt{\delta}$, on obtient : \[E = \mathbb{Q}(\sqrt{\delta})\]
			\end{enumerate}
		\end{proof}


% ----------------------------------------------------
		\section{Ensemble des nombres constructibles}
		
			\subsection{Préliminaires}
			Les opérations de construction à la règle et au compas sont équivalentes à certaines opérations algébriques. Il faut donc formuler un problème algébrique équivalent au problème géométrique avant de conclure.
			
			\begin{definition}
				A chaque étape d'une construction à la régle et au compas, les opérations élémentaires suivantes sont autorisées:
				\begin{enumerate}
					\item Construction d'une intersection de droites définies chacune par deux points construits.
					\item Construction de l'intersection d'une droite définie par deux points construits et d'un cercle défini par deux points construits (son centre et l'extrémité d'un rayon)
					\item Construction d'une intersection entre deux cercles chacun défini par deux points construits.
				\end{enumerate}
			\end{definition}
			
			\begin{definition}
				On définit par récurrence la suite \((\mathscr{C}_i)_{i \in \mathbb{N}} \) :
				
				
				
				\begin{itemize}
					\item \(\mathscr{C}_0 = \{(0, 0), (1, 0)\}\)
					\item \(\forall n \in \mathbb{N}, \mathscr{C}_{n+1} = A\cup \mathscr{C}_{n}\), avec \(A\) l'ensemble des points constructibles par opérations élémentaires à partir de \( \mathscr{C}_{n} \).
				\end{itemize}
			\end{definition}
			
			\begin{definition}
				On définit l'ensemble des points constructibles \(\mathscr{C} = \bigcup_{i \in \mathbb{N}} \mathscr{C}_i \) puis l'ensemble des réels constructibles \(\mathscr{C}_\mathbb{R}\) comme l'ensemble des abscisses des points de \( \mathscr{C} \).
			\end{definition}
				
			
		\begin{proposition} \( \mathscr{C}_\mathbb{R} \) est un sous-corps de \(\mathbb{R}\). A fortiori, on aura donc \( \mathbb{Q} \subset \mathscr{C}_\mathbb{R} \)
		\end{proposition}
			\begin{proof}
				On travaillera dans un repère orthonormé $(O, \vec{OI}, \vec{OJ})$
				\begin{enumerate}
					\item On a $(1, 0) \in \mathscr{C}_0$ par définition, donc $(1, 0) \in \mathscr{C}$ donc $1 \in \mathscr{C}_\mathbb{R}$
					\item Montrons que $\forall (x, y) \in \mathscr{C}_\mathbb{R}, x-y \in \mathscr{C}_\mathbb{R}$. Soit donc $(x, y) \in \mathbb{C}_\mathbb{R}$. Par définition de $\mathscr{C}_\mathbb{R}$, il existe alors deux points constructibles sur l'axe des abscisses $A$ et $B$ d'abscisses respectives $x$ et $y$. Proposons alors une méthode de construction pour le point $F$ de coordonnées $(x-y, 0)$, ce qui suffira à conclure.
						\begin{itemize}
							\item Si $y = 0$ ou $y = 2x$, rien à construire.
							\item Sinon : 
								\begin{enumerate}
									\item Tracer le cercle de centre $O$ passant par $B$. Son intersection avec l'axe des ordonnées est le point $M$.
									\item Tracer les points $C$ et $E$, intersections du cercle de centre $M$ passant par $A$ et du cercle de centre $B$ passant par $M$
									\item Tracer le point $E$, intersection des droites $(CD)$ et $(MA)$
									\item Tracer $F$, l'intersection du cercle de centre $E$ passant par $B$ et de l'axe des abscisses. Le point $F$ a pour coordonnées $(x-y, 0)$. Ce qui achève la construction.	
								\end{enumerate}
								Ici, on a simplement construit un parallélogramme définit par le segment $[AB]$ et le point $M$. Puis on a projeté le sommet construit sur l'axe des abscisses.
						\end{itemize}
					\item Montrons que \( \forall (x, y) \in \mathscr{C}_\mathbb{R}, x \times y \in \mathscr{C}_\mathbb{R}\). Soit donc $(x, y) \in \mathscr{C}_\mathbb{R}$. Supposons donc les points $A (x, 0)$ et $B (y, 0)$ construits. Proposons alors une méthode de construction du point $H$ de coordonnées $(x\times y, 0)$:
						\begin{enumerate}
							\item Construire le point $C$, l'intersection de l'axe des ordonnées avec le point de centre $O$ passant par $B$.
							\item Tracer les points $D$ et $E$, intersection du cercle de centre $A$ passant par $C$ et du cercle de centre $C$ passant par $A$.
							\item Tracer le point $F$, intersection des droites $(DE)$ et $(AC)$
							\item Tracer le point $G$, intersection du cercle de centre $F$ passant par $J$ et de la droite $(JF)$.
							\item Tracer le point $H$, intersection de l'axe des abscisses et de la droite $(CG)$.
						\end{enumerate}
						
					\item Montrons que \( \forall x \in \mathscr{C}_\mathbb{R}\setminus \{0\}, x^{-1} \in \mathscr{C}_\mathbb{R} \). Soit donc $x \in \mathscr{C}_\mathbb{R}$. Sans perte de généralité, quitte à construire $-1/x$, ce qui serait suffisant par le point précédent, supposons $x > 0$. Supposons ainsi le point $A$ de coordonnées $(x, 0)$ construit et construisons le point $G$ de coordonnées $(1/x, 0)$
					\begin{enumerate}
						\item Constuire le point $C$, intersection d'ordonnée positive du cercle de centre $O$ passant par $B$ et de l'axe des ordonnée.
						\item Construire les points $B$ et $D$ intersections du cercle de centre $I$ passant par $J$ et du cercle de centre $J$ passant par $I$.
						\item Constuire le point $E$, intersection des droites $(BD)$ et $(IJ)$.
						\item Construire le point $F$, intersection distincte de $C$ de la $(CE)$ et du cercle de centre $E$ passant par $C$.
						\item Construire $G$ le point d'intersection de $(JF)$ et de l'axe des abscisses.
					\end{enumerate}
					On utilise de même le théorème de Thalès pour conclure directement.
							
				\end{enumerate}
			\end{proof}
		
		\begin{proposition}[Equivalence algèbre-géométrie]
				Les trois opérations élémentaires construisent chacune les racines réelles d'un polynôme de \(\mathbb{K}_2[X]\) dont les coefficients dépendent des points construits au préalable qui servent de base, avec $\mathbb{K}$ le plus petit sous-corps de $\mathscr{C}_\mathbb{R}$ contenant les coordonnées des points construits au préalable.
		\end{proposition}
			\begin{proof}
					On prendra tous nos points de base dans un sous-corps $\mathbb{K}$ de $\mathscr{C}_\mathbb{R}$.
					\begin{enumerate}
						\item{Intersection de deux droites}
							On définit deux droites \( \Delta_1 \), \( \Delta_2 \) définies respectivement à partir des points distincts \(A_1\), \(B_1\) et \(A_2\), \(B_2\), de coordonnées respectives \( (x_{A_1}, y_{A_1}), (x_{B_1}, y_{B_1}), (x_{A_2}, y_{A_2}), (x_{B_2}, y_{B_2})\). On suppose ainsi ces points préalablement construits. On suppose les deux droites non parallèles. Le point d'intersection des deux droites est alors de coordonnées :
							\[
							x = \frac{(x_{A_1} y_{B_1}-y_{A_1} x_{B_1})(x_{A_2}-x_{B_2})-(x_{A_1}-x_{B_1})(x_{A_2} y_{B_2}-y_{A_2} x_{B_2})}{(x_{A_1}-x_{B_1})(y_{A_2}-y_{B_2})-(y_{A_1}-y_{B_1})(x_{A_2}-x_{B_2})}	
							\]
							\[ y=\frac{(x_{A_1}y_{B_1}-y_{A_1}								x_{B_1})(y_{A_2}-y_{B_2})-(y_{A_1}-y_{B_1})(x_{A_2} y_{B_2}-y_{A_2} x_{B_2})}{(x_{A_1}-x_{B_1})(y_{A_2}-y_{B_2})-(y_{A_1}-y_{B_1})(x_{A_2}-x_{B_2})}
							\]		
						\item{Intersection d'une droite et d'un cercle} Soit $\Delta$ une droite définie par les points $A$ et $B$ distincts, et $\mathcal{C}$ le cercle de centre $\Omega$ passant par un point $M$. L'ensemble des intersections est l'ensemble des solutions du système :
							\[
							\left \{
							\begin{array}{ll}
								(y_B - y_A)x + (x_A - x_B)y + x_A(y_A - y_B) + y_A(x_B-x_A) = 0
								\\
								(x-x_\Omega)^2 + (y-y_\Omega)^2 = (x_M - x_\Omega)^2 + (y_M - y_\Omega)^2
							\end{array}
							\right.
							\]
						Puis, $A$ et $B$ étant distincts, on a \(y_B - y_A \neq 0\) ou \(x_B - x_A \neq 0\). Le raisonnement étant symétrique dans les deux cas, supposons que \(x_B - x_A \neq 0\). Alors, en substituant, on obtient :
							\[
							\left \{
							\begin{array}{lllll}
								a = \frac{y_A-y_B}{x_A-x_B}\\
								b = x_A\left(y_B - y_A + \frac{y_B-y_A}{x_A-x_B}\right)+y_A \\
								r = \sqrt{(x_M - x_\Omega)^2 + (y_M - y_\Omega)^2} \\
								y = ax + b \\
								(a^2 + 1)x^2 + 2(a(b-y_B)-x_\Omega)x + {x_\Omega}^2 + (b-y_\Omega)^2 - r^2 = 0
							\end{array}
							\right.
							\]
						Ce qui achève la preuve. D'autre part, s'il existe une solution, elle sera bien, pour chaque coordonnée de l'intersection, de la forme \(k + l \sqrt{\delta_0}\), avec \(k, l, \delta_0 \in \mathbb{K}\).
						\item{Intersection de deux cercles} Si l'intersection existe, une obtient, par le même raisonnement, une même forme de solution.
					\end{enumerate}
					On a ainsi montré qu'une opération de construction revient à la construction d'une \emph{extension quadratique}.
				\end{proof}
		
		\begin{proposition}[Construction d'une racine carrée]
			Etant donné une longueur construite \(l \in \mathbb{R}^+\), il est possible de construire \(\sqrt{l}\).{}
		\end{proposition}
			\begin{proof}
				\begin{enumerate}
					\item{Méthode de construction}
					On se donne au préalable un repère \( (O, \vec{u_x}, \vec{u_y}) \) et un point \(A(l, 0)\). 
						\begin{enumerate}
							\item Tracer le point \(B\) de coordonnées \((-1, 0)\). Pour cela, il suffit de tracer l'intersection du cercle de centre \(O\) passant par \((0, 1)\) avec l'axe des abscisses.
							\item Tracer le milieu \(I\) de \([AB]\). Pour cela, on trace la médiatrice, ce qui nécessite uniquement les opérations élémentaires données par la définition 1.2.1.
							\item L'intersection du cercle de centre \(I\) et de rayon \(IA\) coupe l'axe des ordonnées en \(J\).
							\item \( J \) a alors pour coordonnées \((0, \sqrt{l})\), ce qui achève la construction
						\end{enumerate}
					\item{Validité} On a bien utilisé les opérations élémentaires, et les coordonnées de J sont données par le théorème de Pythagore.
				\end{enumerate}
			\end{proof}
			
		\subsection{Théorème de Wantzel}
			\begin{theorem}
				Un réel \( x \) est constructible si et seulement si il existe une suite finie d'extensions \emph{quadratiques},
					\[
					\mathbb{Q} = K_0 \subset K_1 \subset \dots \subset K_r
					\]
				telle que \( x \in K_r \).
			\end{theorem}
			\begin{proof}
				\begin{enumerate}
					\item{Sens direct}
					Soit \(x \in \mathscr{C}_\mathbb{R}\)
					Alors, par définition, 
					\[\exists j \in \mathbb{N}, x \in \mathscr{C}_j
					\]
					Or, \(\mathscr{C}_0, \dots, \mathscr{C}_j\) est une suite d'extension quadratiques, ce qui nous permet de conclure.
					\item{Sens indirect} Montrons le par récurrence sur r. On sait construire \(\mathbb{Q} = K_0\) par opérations élémentaires. Puis, soit \(r \in \mathbb{N}\) et supposons \(K_r \) construit. Or, \(K_{r+1}\) est une extension quadratique de \(K_{r}\), et on a donc, par la proposition :
					\[
					\exists \delta_0 \in {K_r}^+, K_{r+1} = K(\sqrt{\delta_0})
					\]
					Ainsi, il suffit de construire \(\sqrt{\delta_0}\) pour que \(K_{r+1}\) soit constructible. En effet, on construit nécéssairement un corps contenant \(K_r\) et \(\sqrt{\delta_0}\), donc contenant \(K_{r+1}\) par minimalité de l'extension de corps.
					Or, on sait construire une racine carrée d'un nombre construit, ce qui nous permet de conclure, et d'achever la récurrence.
				\end{enumerate}
			\end{proof}
		La conséquence la plus importante du théorème de Wantzel est donnée par l'énoncé suivant. C'est celui là que l'on utilisera dans la pratique.
			
		    \begin{corollaire}
		     Tout nombre réel constructible est un nombre algébrique dont le degré algébrique est de la forme \( 2^{n}, n\geq 0 \).
		     \end{corollaire}   
		     
		     \begin{proof}
		     Soit x un nombre constructible. Par le théorème de Wantzel, il 
		     existe des extensions quadratiques \( \mathbb{Q} = K_{0} 
		     \subset K_{1} \subset \dots \subset K_{r} \) telles que x 
		     \( \in K_{r} \). Donc x appartient à une extension de \( 
		     \mathbb{Q} \) de degré fini. Ainsi par la proposition 
		     1.1.2, x est algébrique. On sait de plus que \( 
		     [K_{i}:K_{i+1}] = 2 \) donc par la proposition sur 
		     la relation de Chasles sur les degrés \( 
		     [\mathbb{Q}:K_{r}] 
		     = 2^{r} \). Il reste à déduire le degré de \( 
		     [\mathbb{Q}(x):\mathbb{Q}] \). Comme \( \mathbb{Q}(x) \subset K_{r} 
		     \), nous avons toujours par la proposition sur la relation 
		     de Chasles du degré que \( [K{r}:\mathbb{Q}(x)]\times 
		     [\mathbb{Q}(x):\mathbb{Q}] = [K_{r}:\mathbb{Q}] = 2^{r}. 
		     \)Donc\( [\mathbb{Q}(x):\mathbb{Q}] \) divise \(2^{r}\) et est donc 
		     de la forme \(2^{n}\).
		     \end{proof} 
		     
		     \begin{corollaire}
		     
		     \(\mathscr{C}_{\mathbb{R}} \) est le plus petit sous-corps 
		     de \(\mathbb{R}\) stable par racine carréee, c'est-à-dire 
		     telles que :
		     \begin{enumerate}
		      \item \( (x \in \mathscr{C}_{\mathbb{R}}\) et \(x \geq 0) 
		      \implies \sqrt{x} \in \mathscr{C}_{\mathbb{R}}.\)
		      \item Si K est un autre sous-corps de \(\mathbb{R}\) 
		      stable par racine carrée alors \(\mathscr{C}_{\mathbb{R}} 
		      \subset K.\)
		      
		     \end{enumerate}
		     
		     \end{corollaire}
		
		
		\subsection{Exemples célèbres}
		
        On peut désormais répondre aux problèmes de la trissection 
        d'un angle, de la duplication du cube et ainsi que de la 
        quadrature du cercle.
        
			\begin{proposition}
			Si $x\in\mathbb{R}$ est constructible, il est algébrique, et son degré algébrique est de la forme : 
			\[{}
				2^{n}\;\;\mbox{ avec }\;n\in\mathbb{N}
			\]
			\end{proposition}
             
             
             \begin{proposition}
             
             $\sqrt[2]{2}$, indispensable pour dupliquer un cube de côté $1$, n'est pas constructible.
             
             \end{proposition}
             
			\begin{proof}
			En effet, 
				\(\sqrt[3]{2}\) est une racine du polynôme 
				\(P(X)=X^{3}-2\). Ce polynôme est unitaire et 
				irréductible dans \(\mathbb{Q}[X]\), donc \(\sqrt[3]{2}\) 
				est algébrique de degré 3. Ainsi son degré n'est pas de 
				la forme \(2^{n}\). \(\sqrt[3]{2}\) n'est donc pas 
				constructible.

			\end{proof}
		

% --------------------- --------- --------------------- 
% --------------------- PARTIE II -------------------- 
% --------------------- --------- --------------------- 




\chapter{Constructibilité en origami}
	Nous allons voir, au travers de cette partie, comment la formalisation mathématique des pratiques de l'origami donne naissance à une constructibilité étendue, qui sera ensuite utilisée lors de la conception des \emph{crease pattern} de la partie III.

% ----------------------------------------------------
	\section{Définitions et axiomes}
		Même si l'opération du pliage nécessite 3 dimensions, on ne s'interessera qu'au résultat de cet opération : le pli. Ainsi on peut se placer dans le plan muni de sa base canonique, soit $\mathbb{R}^{2}$
	\subsection{Définitions}
		\begin{definition}[Pli et point]
			Un \emph{pli} est une droite de $\mathbb{R}^{2}$, donc en particulier un ensemble de points. Un \emph{point} est un couple de coordonnées réelles.
		\end{definition}
		\begin{definition}[Pliage]
			On dit qu'un pli $P$ \emph{envoie} un point $p_{1}$ sur un autre point $p_{2}$ si $p_{2}$ est l'image de $p_{1}$ par la symétrie d'axe $P$. On dit qu'un pli $P$ \emph{envoie} un pli $P_{1}$ sur un pli $P_{2}$ si tout point de $P_{1}$ s'envoie par $P$ sur $P_{2}$
		\end{definition}
		
	\subsection{Axiomes de Huzita}
		Après avoir posé le support, il est nécessaire de se donner les outils de construction. Ici, pas de règle ni de compas, mais 7 axiomes correspondants aux opérations élémentaires réalisables. On se convaincra aisément que ces opérations sont en effet faisables avec une feuille réelle. \\
		Ce jeu d'axiome n'as pas été montré minimal, et ne l'est peut-être pas. Il est en revanche intuitif, et donne un sens à des opérations naturelles dans la pratique de l'origami. 
		\begin{axiome}[1]
			Étant donné deux points $p_{1}$ et $p_{2}$ distincts, il existe un unique pli qui passe par ces deux points.
		\end{axiome}
		
		\begin{figure}
		    \begin{center}[h]
			\includegraphics[height=130px]{media/Ax1.eps}
		    \end{center}
			\caption{Illustration de l'axiome 1}
		\end{figure}
		
		On peut ainsi construire les droites, comme avec une règle.
		\begin{axiome}[2]
			Étant donné deux points $p_{1}$ et $p_{2}$ distincts, il existe un unique pli qui envoie $p_{1}$ sur $p_{2}$.
		\end{axiome}
		
		\begin{figure}
		    \begin{center}[h]
			\includegraphics[height=130px]{media/Ax2.eps}
		    \end{center}
			\caption{Illustration de l'axiome 2}
		\end{figure}
		
		Cette opération est complétement équivalente au tracage d'une médiatrice entre deux points.
		\begin{axiome}[3]
			Étant donné deux plis $P_{1}$ et $P_{2}$ distincts, il existe un unique pli qui envoie $P_{1}$ sur $P_{2}$.
		\end{axiome}
		
		\begin{figure}
		    \begin{center}[h]
			\includegraphics[height=130px]{media/Ax3.eps}
		    \end{center}
			\caption{Illustration de l'axiome 3}
		\end{figure}
		
		Cette opération est complétement équivalente au tracage d'une bissectrice entre deux droites.
		\begin{axiome}[4]
			Étant donné un point $p_{1}$ et un pli $P_{1}$, il existe un unique pli $P$ qui passe par $p_{1}$ et soit perpendiculaire à $P_{1}$
		\end{axiome}
		
		\begin{figure}
		    \begin{center}[h]
			\includegraphics[height=130px]{media/Ax4.eps}
		    \end{center}
			\caption{Illustration de l'axiome 4}
		\end{figure}
		Cette opération est encore réalisable à la régle et au compas, moyennant la recherche de médiatrices.
		\begin{axiome}[5]
			Étant donné deux points $p_{1}$ et $p_{2}$, et un pli $P_{1}$, il existe un pli $P$ qui envoie $p_{1}$ sur un point de $P_{1}$, en passant par $p_{2}$
		\end{axiome}
		\begin{figure}[h]
		    \begin{center}
			\includegraphics[height=130px]{media/Ax5.eps}
		    \end{center}
			\caption{Illustration de l'axiome 5}
		\end{figure}
		Cette opération correspond à la recherche d'une intersection entre un cercle et une droite. Il peut y avoir 0,1 ou 2 solutions.
		
		
		\begin{axiome}[6]
			Étant donné deux points $p_{1}$ et $p_{2}$, et deux plis $P_{1}$ et $P_{2}$, il existe un pli $P$ qui envoie $p_{1}$ sur un point de $P_{1}$ et $p_{2}$ sur un point de $P_{2}$
		\end{axiome}
		\begin{figure}[h]
		    \begin{center}
			\includegraphics[height=130px]{media/Ax6.eps}
		    \end{center}
			\caption{Illustration de l'axiome 6}
		\end{figure}
		 C'est cette opération précise qui apporte un avantage au système axiomatique de l'origami. En effet on montrera par la suite qu'elle permet de résoudre des équations de degré 3.
		
		
		\begin{axiome}[7]
			Étant donné un point $p_{1}$ et deux plis $P_{1}$ et $P_{2}$, il existe un pli qui envoie $p_{1}$ sur $P_{1}$, perpendiculairement à $P_{2}$
		\end{axiome}
		\begin{figure}[h]
		    \begin{center}
			\includegraphics[height=130px]{media/Ax7.eps}
		    \end{center}
			\caption{Illustration de l'axiome 7}
		\end{figure}
		

% ----------------------------------------------------
\section{Ensemble des nombres origami-constructibles}
			Il est important de traduire concrètement en termes géométriques les axiomes énoncés. Contrairement aux constructions à la règle et au compas, l'origami se concentre plus sur les plis que sur les points.

	
\subsection{Définitions et études des premiers axiomes}

			\begin{definition}
			Un point peut toujours être défini comme l'intersection de deux plis. En plus, si on se donne un ensemble initial de points appelés "points construits", les opérations autorisées sont les suivantes : 
				\begin{enumerate}
					\item Construction d'un pli passant par deux points construits
					\item Construction de la médiatrice de deux points construits
					\item Construction de la bissectrice de deux plis construits
					\item Construction de la perpendiculaire à un pli construit passant par un point contruit.
					\item Étant donné deux points construits $p_{1}$ et $p_{2}$, et un pli construit $P$, si il existe un point d'intersection entre $P$ et le cercle de centre $p_{1}$, passant par $p_{2}$, on peut construire la médiatrice entre $p_{1}$ et ce point d'intersection.
					\item Étant donné deux points construits $p_{1}$ et $p_{2}$, et deux plis construits $P_{1}$ et $P_{2}$, on peut construire la tangente aux paraboles de foyer $p_{1}$ et $p_{2}$, de directrice $P_{1}$ et $P_{2}$
					\item Étant donné un point construit, et deux plis construits $P_{1}$ et $P_{2}$, parmis l'ensemble des médiatrices entre le point construit et un point quelconque de $P_{1}$, on peut construire celle qui est perpendiculaire à $P_{2}$
				\end{enumerate}
			\end{definition}

			\begin{definition}
				On définit par récurrence la suite \((\mathcal{O}_i)_{i \in \mathbb{N}} \) :
				\begin{itemize}
					\item \(\mathcal{O}_0 = \{(0, 0), (1, 0)\}\)
					\item \(\forall n \in \mathbb{N}, \mathcal{O}_{n+1} = A\cup \mathcal{O}_{n}\), avec \(A\) l'ensemble des points constructibles par opérations élémentaires à partir de \( \mathcal{O}_{n} \).
				\end{itemize}
			\end{definition}

			\begin{definition}
				On définit l'ensemble des points constructibles \(\mathcal{O} = \bigcup_{i \in \mathbb{N}} \mathcal{O}_i \) puis l'ensemble des réels constructibles \(\mathcal{O}_\mathbb{R}\) comme l'ensemble des abscisses des points de \( \mathcal{O} \).
			\end{definition}

			\begin{proposition}[Pseudo-équivalence Origami-Règle et compas]
			Les opérations des cinqs premiers axiomes de l'origami sont équivalentes aux opérations à la règle et compas.
			\end{proposition}
			\begin{proof}
			Montrons l'équivalence en construisant les axiomes de l'origami à la règle et au compas, et réciproquement.
			\begin{enumerate}
				\item De l'origami à la règle et au compas.
				\begin{enumerate}
					\item On peut immédiatement construire un point intersection de deux droites\slash plis en origami.
					\item Si on se donne une droite et un cercle, défini par deux points, montrons que l'on peut construire en origami l'intersection.
					Soit $A,B$ les points définissant le cercle, avec $B$ le centre et $D$ la droite. Par l'axiome 5, on peut se donner le pli qui envoie $A$ sur $D$ en passant par $B$, c'est à dire la médiatrice entre $A$ et le point cherché, et on note $E$ ce pli. Il suffit, par l'axiome 2, de prendre la perpendiculaire à $E$ passant par $B$, notée $F$,et son intersection avec $D$ nous donne le point cherché. En effet, si on note $X$ ce point, par propriété des médiatrices, $F$ est parralèle à $(BX)$, et comme $B$ est un point commun, les droites sont confondues.
					\item Si on se donne deux cercles, soit 4 points $A,B,C,D$, avec $A$ et $C$ les centres, montrons que l'on peut construire en origami l'intersection.
				\end{enumerate}
				
				On propose pour chaque axiome de l'origami une construction à la règle et au compas
			\begin{enumerate}
				\item{Premier axiome de Huzita :} \\
					Cet axiome coïncide avec la définition d'une droite, et nous permet donc de décrire un pli origami comme une droite du plan.
				\item{Deuxième axiome de Huzita :}\\
					Cet axiome est équivalent à la construction d'une médiatrice entre deux points. Si l'on se donne un point $A$ et un point $B$ préalablement construits, il suffit de considérer la droite définie à partir des deux intersections du cercle de centre $A$ passant par $B$ et du cercle de centre $B$ passant par $A$.
				\item{Troisième axiome de Huzita :}\\
					Tout revient à construire une bissectrice étant donné deux droites non parallèles.
					
					Soit deux droites $\Delta_1$ et $\Delta_2$ définies respectivement à partir des points $A$ et $B$, et $C$ et $D$ préalablement construits. Soit $O$ l'intersection des deux droites. On définit $A'$ comme l'intersection de $\Delta_2$ avec le cercle de centre $O$ passant par $A$. La médiatrice du segment $[AA']$ correspond au pli construit par le troisième axiome de Huzita.
				
				\item{Quatrième axiome de Huzita :}\\
					On se donne ici une droite $\Delta$ définie par les points $A$ et $B$ préalablement construits, et un point $M$, distinct de la droite. Considérons les intersections du cercle de centre $M$ passant par $A$ avec $\Delta$. S'il n'en existe qu'une, $A$, et alors le cercle est tangent à la droite, donc la droite $(AM)$ convient. Sinon, en notant $A'$ la seconde intersection, la médiatrice du segment $[AA']$ convient.
				\item{Cinquième axiome de Huzita :}\\
					Etant donné deux points $A$ et $B$ et une droite $\Delta$ définie par deux points $C$ et $D$, il suffit de considérer la médiatrice du segment défini par $B$ et l'intersection du cercle de centre $A$ passant par $B$ avec la droite $\Delta$.
				\end{enumerate}
			\end{enumerate}
			\end{proof}

\subsection{La spécificité de l'axiome 6}
			
			\begin{definition}[Parabole]
				On se place dans un plan $P$. Soit $D$ une droite et $F$ un point n'appartenant pas à la droite. La parabole de foyer $F$ et de directrice $D$ est l'ensemble des points à égale distance de de $F$ et de $D$. Formellement, si pour $X \in P$, on note $H_{X}$ le projeté ortogonal de $X$ sur $D$, la parabole est exactement  
				\[\{X\in P /\; XF = XH_{X}\}\]
			\end{definition}
			\begin{definition}[Tangente à une parabole]
				Soit $\mathcal{P}$ une parabole de foyer $F$ et de directrice $D$. Soit $A$ un point de $\mathcal{P}$. La tangente en $A$ à la parabole est la bissectrice interieure de l'angle $\widehat{AFH_{A}}$.
			\end{definition}
			
			\paragraph{Le problème de l'axiome 6}
			Deux définitions, qui semblent proches, coexistentent, mais ont des conséquences différentes. La version forte de l'axiome est celle que nous avons énoncé. La version faible de l'axiome est le sujet de cette sous-partie, en voici l'énoncé :
			
			\begin{axiome}[6 - Version faible]
			Étant donné deux points $p_{1}$ et $p_{2}$, et deux plis $P_{1}$ et $P_{2}$, tel que $p_{1}$ ne soit pas un point de $P_{1}$ et $p_{2}$ ne soit pas un pli de $P_{2}$, alors il existe un pli $P$ qui envoie $p_{1}$ sur un point de $P_{1}$ et $p_{2}$ sur un point de $P_{2}$
			\end{axiome}
			
			Alors, on comprends que les couples $(p_{1},P_{1})$ et $(p_{2},P_{2})$ pourront servir à définir des paraboles, et c'est en effet le sujet de la proposition suivante : 
			
			\begin{proposition}
				Soit deux points $p_{1},p_{2}$ et deux plis (droites) $P_{1},P_{2}$. On pose $\mathcal{P}_{1},\mathcal{P}_{2}$ les paraboles de foyer respectif $p_{1}$ et $p_{2}$, de directrice respective $P_{1}$ et $P_{2}$. Se donner le pli envoyant $p_{1}$ sur $P_{1}$ et $p_{2}$ sur $P_{2}$ par l'axiome 6 (version faible) est équivalent à chercher une droite qui soit tangente à la fois de $\mathcal{P}_{1}$ et $\mathcal{P}_{2}$
			\end{proposition}
			
			\begin{proof}
				La preuve se fait en deux temps:
				\begin{enumerate}
					\item Soit $\mathcal{P}$  une parabole de foyer $F$ et de directrice $D$. Montrons que les tangentes à $\mathcal{P}$ sont exactement les plis qui envoient $F$ sur $\mathcal{D}$. Pour $A$ un point quelconque, on notera $H_{A}$ le projeté orthogonal de $A$ sur $D$.
					\begin{enumerate}
						\item Soit $T$ une tangente à $\mathcal{P}$ en $A$. Le triangle $AFH_{A}$ est isocèle en $A$, et la bissectrice de l'angle directeur $\widehat{AFH_{A}}$. Or dans un triangle isocèle, la bissectrice de l'angle directeur est confondue avec la médiatrice de la base. Donc $T$ est la médiatrice de $F$ et $H_{A}$, donc vue comme pli, elle envoie $F$ sur $H_{A}$, qui est bien un point de $\mathcal{D}$
						
						\item Soit $T$ un pli qui envoie $F$ sur un point de $\mathcal{D}$, noté $A$. C'est en particulier une médiatrice de $F$ et $A$. Considérons la perpendiculaire à $\mathcal{D}$ passant par $A$, notée $\mathcal{E}$. On remarque que $\mathcal{D}$ et $\mathcal{E}$ ne sont jamais parralèles, sans quoi $F$ appartiendrait à $\mathcal{D}$, ce qui n'est pas le cas, par hypothèse. Donc $\mathcal{D}$ et $\mathcal{E}$ admettent un point d'intersection, que l'on note $B$. Ce point est à même distance de $A$ et de $F$, et même, comme il appartient à $\mathcal{E}$, il est à la même distance de $\mathcal{D}$ et de $F$, donc appartient à $\mathcal{P}$. Par propriété des triangles isocèles, $T$ est bien la médiatrice interieure de $FBA$, donc une tangente à $\mathcal{P}$ en $B$.
					\end{enumerate}
					
					\item Ainsi, on a directement l'equivalence entre trouver un pli qui envoie $p_{1}$ sur $P_{1}$ et $p_{2}$, sur $P_{2}$, qui sera en effet une tangente commune à $\mathcal{P}_{1}$ et $\mathcal{P}_{2}$, et se donner réciproquement une tangente commune, qui enverra $p_{1}$ sur $P_{1}$ et $p_{2}$, sur $P_{2}$.
				\end{enumerate}
			\end{proof}
		Par la suite, on verra que utiliser la version forte de l'axiome 6 est équivalent à utiliser conjointement l'axiome 7 et la version faible de l'axiome 6. Mais, sachant que l'équivalence algébrique est alors bien plus rapide, on comprends l'utilité de cette approche.
	
\subsection{La redondance de l'axiome 7}
		
		\paragraph{} En continuant à progresser dans les caractéristiques du système axiomatique de Huzita, on en vient à remarquer que ce système est redondant, en l'occurence, que l'axiome 7 peut être construit en utilisant les autres.
		
		\begin{proposition}[Construction de l'axiome 7]
			Étant donné un point $p_{1}$ et deux plis $P_{1}$ et $P_{2}$, il est possible de construire un pli qui envoie $p_{1}$ sur $P_{1}$, perpendiculairement à $P_{2}$, en utilisant les 4 premiers axiomes seulement
		\end{proposition}
		
		
		\begin{proof}
			Plaçons nous d'abord dans les condititions de faisabilité de l'axiome 7, c'est à dire quand $P_{1}$ et $P_{2}$ ne sont pas parralèles : en effet, les plis perpendiculaires à $P_{2}$ envoient nécessairement $p_{1}$ sur une parralèle de $P_{2}$, et en excluant le cas dégénéré où $p_{1}$ est déjà sur $P_{1}$, aucun pli ne peut convenir. On suppose ainsi que $P_{1}$ et $P_{2}$ ne sont pas parralèles. Voici le processus de construction : 
			\begin{itemize}
				\item Construire (axiome 4) la perpendiculaire à $P_{2}$ passant par $p_{1}$, notée $A$.
				\item Construire (axiome 4) la perpendiculaire à $A$ passant par $p_{1}$, notée $B$. Cette droite est parallèle à $P_{2}$, car $A$ est perpendiculaire à $A$ et $P_{2}$.
				\item Ainsi il existe un point d'intersection entre $B$ et $P_{1}$, noté $\delta$. Construire (axiome 2) la médiatrice entre $p_{1}$ et $\delta$, notée $\Delta$.
			\end{itemize}
			Comme perpendiculaire à $B$, $\Delta$ est perpendiculaire aussi à $P_{2}$. De plus, comme médiatrice, $\Delta$ envoie $p_{1}$ sur $\delta$, qui est un point de $P_{1}$, donc convient comme pli pour l'axiome 7.
		\end{proof}
		
		On pourrait alors s'interroger sur l'utilité de cet axiome, sachant la simplicité de son "émulation" en utilisant les autres axiomes. Il y a deux aspects au problème. Premièrement, cet axiome, comme tous les autres, a pour vocation de représenter géométriquement une construction qu'il est naturel de faire, feuille en main. L'oppération correspondante est certe redondante, mais peut-être faite en "une fois" lorsque l'on manipule la feuille directement, et, en cela, l'axiome représente fidèlement la réalité du pliage.\\
		Deuxièmement, l'axiome 7 permet de traiter un cas particulier de l'axiome 6, où des deux points, par exemple $p_{1}$, est déjà sur la droite sur laquelle il doit s'envoyer, ici $P_{1}$ du coup. Le pli sera alors perpendiculaire à $P_{1}$, envoyant $p_{2}$ sur $P_{2}$. On retrouve l'axiome 7. Le dernier cas dégénéré (points et droites confondues) pouvant être traité de la même façon avec les 5 premiers axiomes, on en déduit le corrolaire suivant :
		
		\begin{corollaire}
			La structure axiomatique de Huzita est équivalente à la structure restreinte aux 6 premiers axiomes, avec le 6 axiome dans sa version forte. Cette structure est aussi équivalente à la structure complète, mais avec le 6 axiome dans sa version faible.
		\end{corollaire}
		
% ----------------------------------------------------
	\section{Extension du théorème de Wantzel}
	
	\begin{proposition}[Equivalence algèbre-géométrie]
				Les trois opérations élémentaires construisent chacune les racines réelles d'un polynôme de \(\mathbb{K}_3[X]\) dont les coefficients dépendent des points construits au préalable qui servent de base, avec $\mathbb{K}$ le plus petit sous-corps de $\mathcal{O}_\mathbb{R}$ contenant les coordonnées des points construits au préalable.
	\end{proposition}
	
	\begin{proof}
		à faire...
	\end{proof}
	
	
	\subsection{Extension du théorème de Wantzel}
		\begin{theorem}
			Un réel \( x \) est origami-constructible si et seulement si il existe une suite finie d'extensions qui chacune soit au plus \emph{cubique},
					\[
					\mathbb{Q} = K_0 \subset K_1 \subset \dots \subset K_r
					\]
			telle que \( x \in K_r \).
		\end{theorem}
	
	\subsection{Exemples célèbres revisités}
		\subsubsection{La trisection d'Abe}
		\subsubsection{La duplication du cube}

% ----------------------------------------------------
	\section{Hyperorigami et nombres hyper-constructibles}
	





% --------------------- ---------- --------------------- 
% --------------------- PARTIE III -------------------- 
% --------------------- ---------- --------------------- 




	
\chapter{Le \emph{crease pattern} et la faisabilité}

Après avoir réalisé le modèle, on déplie la feuille. On peut alors 
observer différents plis qui constituent le canevas de 
"l'origami" (\emph{crease pattern} en anglais). Seuls les plis finaux 
sont à considérer dans l'élaboration du 
canevas, c'est-à-dire ceux qui interviennent dans la structure finale 
du modèle. 


% ----------------------------------------------------
	\section{Définitions et exemples}
 
    \begin{definition} 
      
      Le canevas de pli est la représentation du carré de départ ainsi que l'ensemble des plis 
      servant à la base départ du modèle.
      
    \end{definition}
    
    Par la suite, il sera important de considérer trois types de plis.
    
      \begin{definition}
      
        Après dépliage de la feuille, on peut observer trois 
        catégories de plis:
        
        \begin{enumerate}
        
          \item Les plis \emph{montagnes} qui sont les plis correspondant à 
          une arrête. Il seront en rouge.
          \item Les plis \emph{vallées} qui correspondent aux creux. Ils 
          seront en bleu.
          \item Les plis intermédiaires intervenant lors des étapes 
          précedentes de la construction. Ils seront laissés 
          volontairement sans couleur.
        
        \end{enumerate}
      
      \end{definition}
    
    
    

% ----------------------------------------------------
  \section{De l'origami au \emph{crease pattern} : les conditions 
  nécessaires}  
  
    \paragraph{} Il serait faux de penser qu’avec n’importe quel schéma de plis bleus et rouges on pourrait obtenir un canevas de figure. C’est pour cela que des théorèmes vont permettre dans un premier de temps de vérifier la validité du canevas obtenu.
  
    
    
    \paragraph{} Dans cette partie, le nombre de traits rouges du canevas sera noté \(\mathscr{N}_{R}\) et le nombre de traits bleus sera noté \(\mathscr{N}_{B}\).
    
    \subsection{Théorème de Meakawa}
    
      \begin{theorem}
        Dans un cannevas de plis, les nombres de traits rouges et  de traits bleus se rencontrant en un point intérieur à la feuille doivent vérifer : \(|\mathscr{N}_{R}-\mathscr{N}_{B}|=2\)
       
      \end{theorem}
    
      \begin{proof}
      \end{proof}
    
      Cette propriété de pliage énonce une condition absolument nécessaire pour vérifier si un canevas est juste ou non.
      Découle de ce théorème trois propriétés propres aux plis du canevas.
      
    \subsection{Conséquences du Théorème de Meakawa}
    
      \begin{proposition}
        
        Les zone délimitées par les plis \emph{montagnes} ou \emph{vallées} sont coloriables de deux couleurs sans que deux zones voisines n’aient jamais la même couleur.
      
      \end{proposition} 
      
      \begin{proof}
      \end{proof}
      
      
      \begin{proposition}
      
        Tout sommet du graphe dessiné par les plis \emph{vallées} et 
        les plis \emph{montagnes} vérifient le Théorème de Meakawa.
      
      \end{proposition}
      
      \begin{proof}
      \end{proof}
      
      \begin{remarque}
      
      Tout canevas est impossible si ces propriétés ne sont pas 
      vérifiées. Cependant cela n'est pas suffisant. En effet tout 
      canevas qui possède ces propriétés n'est pas pour autant 
      pliable selon ses plis.
      
      
      \end{remarque}
      
      \begin{theorem}[Théorème de Kawasaki]
      
      \end{theorem}
    
    


% --------------------- --------- --------------------- 
% --------------------- PARTIE VI -------------------- 
% --------------------- --------- --------------------- 




\chapter{La conception assistée : étude du logiciel TreeMaker}


\end{document}
